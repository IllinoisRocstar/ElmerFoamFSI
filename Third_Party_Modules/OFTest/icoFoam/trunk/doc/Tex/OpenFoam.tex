\irsection{OpenFoam}{OpenFoam}

In order to build a module for \icofoam\, it is necessary to first download and build \openfoam\, and specifically for this case \openfoamex\, (\openfoam\, and \openfoamex\, will be used interchangeably within this document).  Information for downloading this software can be found at \url{http://www.extend-project.de/}.

\textbf{Note:} The same \software{mpicc}, \software{mpicxx}, and/or \software{mpif90} must be used when compiling \software{IMPACT}, \openfoam, and the module driver. 

\jek{Is the above correct, Mike C.? I couldn't remember if I had set those variables before compiling \openfoam\, but I assume I did}

For builds outside of Illinois Rocstar make note of the compiler used to build \openfoam\, and then follow the build instructions for \openfoam\, found online or in the documentation. If you are working within Illinois Rocstar then load the \irfilename{openmpi-x86\_64} module and set the environment variables \irfilename{CC}, \irfilename{CXX}, and \irfilename{FC} to \irfilename{mpicc}, \irfilename{mpicxx}, and \irfilename{mpif90} respectively. Then, source the appropriate file (\irfilename{cshrc} when using \software{C Shell} or \irfilename{bashrc} when using \software{Bash}) found in the \irfilename{etc} directory under the main \software{OpenFoam} source directory by entering one of the following commands

\commandline{source etc/cshrc} 

or
 
\commandline{source etc/bshrc}

from the main \openfoam\, source directory. Finally, run

\commandline{wmake/wmake all} 

from the main \openfoam\, source directory. There may be additional changes that must be made to the \irfilename{wmake} files in order to get \openfoam\, to properly compile within your specific environment setup. 

After \openfoam\, is built, ensure that the \irfilename{icofoam} executable has been created. If you have sourced one of the files for \openfoam\, as mentioned above you may check that the \irfilename{icofoam} executable exists by running 

\commandline{which icofoam}

and ensuring that the \irfilename{icofoam} command returned is within the path of the \openfoam\, source files you have been building (\openfoam\, builds in-source).

Within Illinois Rocstar, a test case exists for \icofoam\, from \openfoamex. The documentation for this test case can be found in Section 9.2.2 of the \irfilename{Uniphysics Validation for Development of Multiphysics Coupling in MP-Infra} document located internally in the Illinois Rocstar source repository under \irfilename{/MPInfra/data/documentation/validation\_uniphysics/Tex}. After ensuring that the test run works, the next steps toward building a module can be taken.
